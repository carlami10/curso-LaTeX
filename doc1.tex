%Clase de documento
\documentclass[a4paper,12pt]{article}

% Paquetes
\usepackage[utf8]{inputenc}
\usepackage[spanish]{babel}
\usepackage[total={18cm, 21cm}, top=2cm, left=2cm]{geometry}
\usepackage{amsmath, amssymb, amsfonts, latexsym}
\usepackage{graphicx}
\usepackage{color}


% Comandos
%sangría
\parindent =0mm
\author{Carla Miranda}
\title{Documento 1}
\date{4 de Abril de 2017}

% Preambulo 
% Contenido
\begin{document}
	\maketitle
	
empezar a escribir
		
salto de línea 
		
\section[EPN]{Escuela Politecnica Nacional}
\subsection[EDO]{título completo}

esta la cabra


Porque me da la gana to\ tal.

\centerline{Lea esta frase, por favor.}

\begin{center}
	<<esto es lo que quiero que haga>>
	\emph{woody allen}
\end{center}

\textbf{hola 2}

{\LARGE Cogito, ergo sum}

\emph{Algo de texto negro, \color{red}
seguido por un fragmento rojo}, {\color
{magenta} finalmente algo de texto magenta.}

%texto a la derecha
\begin{flushright}
There are two types of people in this world, good and bad. \\
The good sleep better, but the bad seem to enjoy \\
the waking hours much more. \\
\textbf{Woody Allen (director neoyorquino)}
\end{flushright}

%espacios horizontales
%\hspace{Longitud}

% Listas numeracion 
Aristoteles pensaba que hay tres clases de felicidad:
\begin{itemize}
\item La felicidad de quien vive de diversiones y placeres.
\item La felicidad de quien vive como ciudadano libre y responsable.
\item La felicidad de quien vive como filosofo y pensador.
\end{itemize}
Pensaba que era verdaderamente feliz solo quien


\end{document}
