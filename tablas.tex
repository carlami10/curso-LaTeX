\documentclass[10pt,a4paper]{article}
\usepackage[utf8]{inputenc}
\usepackage{amsmath}
\usepackage{amsfonts}
\usepackage{amssymb}
\usepackage{makeidx}
\usepackage{graphicx}
\usepackage[left=2cm,right=2cm,top=2cm,bottom=2cm]{geometry}
\usepackage{multirow}
\usepackage[x11names,table]{xcolor}
\usepackage{longtable}
\usepackage{booktabs}
\usepackage{url}

\author{carla}
\title{tablas}
\definecolor{db}{RGB}{0,128,128}
\definecolor{dh}{RGB}{128,128,0} %dh



\begin{document}


	\begin{tabular}{cccc}

Primera & Segunda & Tercera & Cuarta 	\\\hline
A 		& B 	& 			& C 		\\\hline
		& D 	& E 		& F 		\\\hline
G 		& 		& 			& H
	\end{tabular}


	\begin{center}
	\begin{tabular}{ | l | l | l | p{5cm} |}
% p{5cm}: Quiere decir que me escriba en un parrafo de 5cm
\hline
Day & Min Temp & Max Temp & Summary \\ \hline
 Monday & \(11\)C & \(22\)C & A clear day with lots of sunshine.
However, the strong breeze will bring down the temperatures. \\ \hline
Tuesday & \(9\)C & \(19\)C & Cloudy with rain, across many northern regions. Clear spells
across most of Scotland and Northern Ireland,
but rain reaching the far northwest. \\ \hline
Wednesday & \(10\)C & \(21\)C & Rain will still linger for the morning.
Conditions will improve by early afternoon and continue
throughout the evening. \\
	\hline
	\end{tabular}
	\end{center}

\begin{center}
\begin{tabular}{|r @{-pablo-} l|} %Para alinear y poner lo que quiero eque este entre la mitad {.}
\hline
\(1\) & \(4\) \\ \hline
\(2\) & \(54\) \\ \hline
\(1\) & \(7890\) \\ \hline
\end{tabular}
\end{center}


\begin{tabular}{|l|l|c|}\hline
\multicolumn{3}{|c|}{PART\'{I}CULAS AT\'{O}MICAS ELEMENTALES} \\
\hline \hline
\textsf{Part\'{i}cula} & \textsf{Descubridor} & \textsf{A\~{n}o del descubrimiento} \\
\hline
Electr\'{o}n & Joseph J. Thomson* & 1897 \\
\hline
Prot\'{o}n & James Rutherford & 1919 \\
\hline
Neutr\'{o}n & James Chadwick* & 1932 \\
\hline
Positr\'{o}n & Carl D. Anderson* & 1932 \\
\hline
\multicolumn{2}{l}{\small *Recibi\'{o} el premio Nobel}
\end{tabular}


\begin{center}
\begin{tabular}{ |l|l|l| }
\hline
\multicolumn{3}{ |c| }{Team sheet} \\
\hline
Goalkeeper & GK & Paul Robinson \\ \hline
\multirow{4}{*}{Defenders} & LB & Lucus Radebe \\ %\multirow: para combinar filas
& DC & Michael Duburry \\
& DC & Dominic Matteo \\
& RB & Didier Domi \\
\hline
\multirow{3}{*}{Midfielders} & \multirow{3}{*}{MC} & David Batty \\
& & Eirik Bakke \\
& & Jody Morris \\
\hline
Forward & FW & Jamie McMaster \\ \hline
\multirow{2}{*}{Strikers} & \multirow{2}{*}{ST} & Alan Smith \\
& & Mark Viduka \\
\hline
\end{tabular}
\end{center}

\begin{table}[h]
\centering
% 


\begin{tabular}{|c|c|c|} \hline
\(p\) & \(q\) & \(p \to q\) \\
\hline
0 & 0 & 1 \\
0 & 1 & 1 \\
\cline{1-2}
1 & 0 & 0 \\
1 & 1 & 1 \\
\hline
\end{tabular}
\caption{Tabla de verdad para \(p \to q\).}
\label{tabla:01}
\end{table}

\begin{center}
\begin{tabular}{| >{\columncolor{db!20}} c | c | >{\color{magenta}} c |}
\hline
\rowcolor{db}
\color{yellow!30!white} \textbf{Temperatura} &
\color{yellow!30!white} \textbf{Tiempo} &
\color{yellow!30!white} \textbf{Total} \\ \hline
1 & \(245 \pm 5.5\) & 3 \\ \hline
2 & \(260 \pm 5.5\) & \cellcolor{white!70!dh}\color{dh}{8} \\ \hline % para pintar una sola
	celda
3 & \(275 \pm 5.5\) & 8 \\ \hline
4 & \(287 \pm 5.5\) & 8 \\ \hline
\end{tabular}
\end{center}

\begin{longtable}{lll}
\caption{Lista de Estudiantes} \\
\toprule
\textbf{Nombre} & \textbf{Carrera} & \textbf{Correo electrónico} \\
\midrule
\endfirsthead
%
\multicolumn{3}{l}{\footnotesize Viene de la página anterior} \\
\toprule
\textbf{Nombre} & \textbf{Carrera} & \textbf{Correo Electrónico} \\ \midrule
\endhead
\bottomrule \multicolumn{3}{r}{\footnotesize Continua en la siguiente página}
\endfoot
\bottomrule
\endlastfoot
%
Milton Torres & Matemática & \url{mate6666oz@hotmail.com} \\
\end{longtable}




\end{document}



